\documentclass{ctexart}
    \usepackage{listings}
    \usepackage{geometry}
    \geometry{a4paper,scale=0.8}
    \usepackage{xcolor}
    \lstset{
        numbers=left, 
        numberstyle= \tiny, 
        showstringspaces=false,%不显示代码字符串中间的空格标记
        showspaces=false,
        showtabs=false,  
        keepspaces=true, %
        breakindent=22pt, %
        stringstyle=\ttfamily, % typewriter type for strings
        basicstyle=\footnotesize, % print whole listing small
        breaklines=true, %对过长的代码自动换行
        breakautoindent=true,%
        breakindent=4em, %
        keywordstyle= \color{ blue!70},
        commentstyle= \color{red!50!green!50!blue!50}, 
        frame=shadowbox, % 阴影效果
        rulesepcolor= \color{ red!20!green!20!blue!20} ,
        escapeinside=``, % 英文分号中可写入中文
        xleftmargin=2em,xrightmargin=2em, aboveskip=1em,
        framexleftmargin=2em
    }
\begin{document}
\lstset{language=bash}
\begin{lstlisting}
#!/usr/bin/evn bash

SYSTEM_MOUNT_POINT=/tmp/houdini/system
BRIDGE_UNZIP_FOLDER=/tmp/houdini/bridge
RAMDISK_UNPACK_FOLDER=/tmp/houdini/ramdisk

# TODO adjust system.img file size to make space for houdini files
mkdir -p ${SYSTEM_MOUNT_POINT}
mkdir -p ${BRIDGE_UNZIP_FOLDER}

# mount system.img and houdini file
mount -t ext4 -o loop system.img ${SYSTEM_MOUNT_POINT}
if [ $? -eq 0 ]; then
    echo "mount system.img success"
else
    echo "mount system.img fail"
fi
mount -t squashfs -o loop houdini.sfs ${BRIDGE_UNZIP_FOLDER}
if [ $? -eq 0 ]; then
    echo "mount houdini.sfs success"
else
    echo "mount houdini.sfs fail"
    umount ${SYSTEM_MOUNT_POINT}
fi

# copy houdini files to /system/lib/arm
mkdir ${SYSTEM_MOUNT_POINT}/lib/arm
cp -R ${BRIDGE_UNZIP_FOLDER}/* ${SYSTEM_MOUNT_POINT}/lib/arm

# umount houdini file and remove temp files
umount ${BRIDGE_UNZIP_FOLDER}
rm -rf ${BRIDGE_UNZIP_FOLDER}

error_exit() {
    umount ${SYSTEM_MOUNT_POINT}
    rm -rf ${BRIDGE_UNZIP_FOLDER}
    exit 1
}

set_perm() {
    chown -R $2:$3 $1 || error_exit
    chmod -R $4 $1 || error_exit
}

# chown chmod
set_perm ${SYSTEM_MOUNT_POINT}/lib/arm 0 0 0644
chmod 0755 ${SYSTEM_MOUNT_POINT}/lib/arm ${SYSTEM_MOUNT_POINT}/lib/arm/nb

# move /system/lib/arm/libhoudini.so -> /system/lib/libhoini.so
# move /system/lib/arm/houdini -> /system/bin/houdini.so
mv ${SYSTEM_MOUNT_POINT}/lib/arm/libhoudini.so ${SYSTEM_MOUNT_POINT}/lib/libhoudini.so
mv ${SYSTEM_MOUNT_POINT}/lib/arm/houdini ${SYSTEM_MOUNT_POINT}/bin/houdini

set_perm ${SYSTEM_MOUNT_POINT}/bin/houdini 0 2000 0755

# modify build.prop
echo "ro.dalvik.vm.isa.arm=x86" >>${SYSTEM_MOUNT_POINT}/build.prop
echo "ro.enable.native.bridge.exec=1" >>${SYSTEM_MOUNT_POINT}/build.prop
sed -i "s/^ro.product.cpu.abi=x86/&\nro.product.cpu.abi2=armeabi-v7a/" ${SYSTEM_MOUNT_POINT}/build.prop
sed -i "s/^ro.product.cpu.abilist=x86/ro.product.cpu.abilist=x86,armeabi,armeabi-v7a/" ${SYSTEM_MOUNT_POINT}/build.prop
sed -i "s/^ro.product.cpu.abilist32=x86/ro.product.cpu.abilist32=x86,armeabi,armeabi-v7a/" ${SYSTEM_MOUNT_POINT}/build.prop

# umount system.img
umount ${SYSTEM_MOUNT_POINT}
rm -rf ${SYSTEM_MOUNT_POINT}

# modify ramdisk.img
MAIN_FOLDER=$(pwd)
mkdir -p ${RAMDISK_UNPACK_FOLDER}
cp ramdisk.img ${RAMDISK_UNPACK_FOLDER}/ramdisk.img.gz
cd ${RAMDISK_UNPACK_FOLDER}
gunzip ramdisk.img.gz
cpio -i -F ramdisk.img
sed -i "s/ro.dalvik.vm.native.bridge=0/ro.dalvik.vm.native.bridge=libhoudini.so/" default.prop
cpio -i -t -F ramdisk.img >list
cpio -o -H newc -O new_ram.img <list
gzip new_ram.img
mv new_ram.img.gz ${MAIN_FOLDER}/ramdisk_new.img

cd ${MAIN_FOLDER}
rm -rf ${RAMDISK_UNPACK_FOLDER}

echo "Done"
\end{lstlisting}
\end{document}\end{document}